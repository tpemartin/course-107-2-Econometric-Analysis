\documentclass[]{article}
\usepackage{lmodern}
\usepackage{amssymb,amsmath}
\usepackage{ifxetex,ifluatex}
\usepackage{fixltx2e} % provides \textsubscript
\ifnum 0\ifxetex 1\fi\ifluatex 1\fi=0 % if pdftex
  \usepackage[T1]{fontenc}
  \usepackage[utf8]{inputenc}
\else % if luatex or xelatex
  \ifxetex
    \usepackage{mathspec}
  \else
    \usepackage{fontspec}
  \fi
  \defaultfontfeatures{Ligatures=TeX,Scale=MatchLowercase}
\fi
% use upquote if available, for straight quotes in verbatim environments
\IfFileExists{upquote.sty}{\usepackage{upquote}}{}
% use microtype if available
\IfFileExists{microtype.sty}{%
\usepackage{microtype}
\UseMicrotypeSet[protrusion]{basicmath} % disable protrusion for tt fonts
}{}
\usepackage[margin=1in]{geometry}
\usepackage{hyperref}
\hypersetup{unicode=true,
            pdftitle={R for OLS},
            pdfauthor={lin-tingyu},
            pdfborder={0 0 0},
            breaklinks=true}
\urlstyle{same}  % don't use monospace font for urls
\usepackage{color}
\usepackage{fancyvrb}
\newcommand{\VerbBar}{|}
\newcommand{\VERB}{\Verb[commandchars=\\\{\}]}
\DefineVerbatimEnvironment{Highlighting}{Verbatim}{commandchars=\\\{\}}
% Add ',fontsize=\small' for more characters per line
\usepackage{framed}
\definecolor{shadecolor}{RGB}{248,248,248}
\newenvironment{Shaded}{\begin{snugshade}}{\end{snugshade}}
\newcommand{\KeywordTok}[1]{\textcolor[rgb]{0.13,0.29,0.53}{\textbf{#1}}}
\newcommand{\DataTypeTok}[1]{\textcolor[rgb]{0.13,0.29,0.53}{#1}}
\newcommand{\DecValTok}[1]{\textcolor[rgb]{0.00,0.00,0.81}{#1}}
\newcommand{\BaseNTok}[1]{\textcolor[rgb]{0.00,0.00,0.81}{#1}}
\newcommand{\FloatTok}[1]{\textcolor[rgb]{0.00,0.00,0.81}{#1}}
\newcommand{\ConstantTok}[1]{\textcolor[rgb]{0.00,0.00,0.00}{#1}}
\newcommand{\CharTok}[1]{\textcolor[rgb]{0.31,0.60,0.02}{#1}}
\newcommand{\SpecialCharTok}[1]{\textcolor[rgb]{0.00,0.00,0.00}{#1}}
\newcommand{\StringTok}[1]{\textcolor[rgb]{0.31,0.60,0.02}{#1}}
\newcommand{\VerbatimStringTok}[1]{\textcolor[rgb]{0.31,0.60,0.02}{#1}}
\newcommand{\SpecialStringTok}[1]{\textcolor[rgb]{0.31,0.60,0.02}{#1}}
\newcommand{\ImportTok}[1]{#1}
\newcommand{\CommentTok}[1]{\textcolor[rgb]{0.56,0.35,0.01}{\textit{#1}}}
\newcommand{\DocumentationTok}[1]{\textcolor[rgb]{0.56,0.35,0.01}{\textbf{\textit{#1}}}}
\newcommand{\AnnotationTok}[1]{\textcolor[rgb]{0.56,0.35,0.01}{\textbf{\textit{#1}}}}
\newcommand{\CommentVarTok}[1]{\textcolor[rgb]{0.56,0.35,0.01}{\textbf{\textit{#1}}}}
\newcommand{\OtherTok}[1]{\textcolor[rgb]{0.56,0.35,0.01}{#1}}
\newcommand{\FunctionTok}[1]{\textcolor[rgb]{0.00,0.00,0.00}{#1}}
\newcommand{\VariableTok}[1]{\textcolor[rgb]{0.00,0.00,0.00}{#1}}
\newcommand{\ControlFlowTok}[1]{\textcolor[rgb]{0.13,0.29,0.53}{\textbf{#1}}}
\newcommand{\OperatorTok}[1]{\textcolor[rgb]{0.81,0.36,0.00}{\textbf{#1}}}
\newcommand{\BuiltInTok}[1]{#1}
\newcommand{\ExtensionTok}[1]{#1}
\newcommand{\PreprocessorTok}[1]{\textcolor[rgb]{0.56,0.35,0.01}{\textit{#1}}}
\newcommand{\AttributeTok}[1]{\textcolor[rgb]{0.77,0.63,0.00}{#1}}
\newcommand{\RegionMarkerTok}[1]{#1}
\newcommand{\InformationTok}[1]{\textcolor[rgb]{0.56,0.35,0.01}{\textbf{\textit{#1}}}}
\newcommand{\WarningTok}[1]{\textcolor[rgb]{0.56,0.35,0.01}{\textbf{\textit{#1}}}}
\newcommand{\AlertTok}[1]{\textcolor[rgb]{0.94,0.16,0.16}{#1}}
\newcommand{\ErrorTok}[1]{\textcolor[rgb]{0.64,0.00,0.00}{\textbf{#1}}}
\newcommand{\NormalTok}[1]{#1}
\usepackage{longtable,booktabs}
\usepackage{graphicx,grffile}
\makeatletter
\def\maxwidth{\ifdim\Gin@nat@width>\linewidth\linewidth\else\Gin@nat@width\fi}
\def\maxheight{\ifdim\Gin@nat@height>\textheight\textheight\else\Gin@nat@height\fi}
\makeatother
% Scale images if necessary, so that they will not overflow the page
% margins by default, and it is still possible to overwrite the defaults
% using explicit options in \includegraphics[width, height, ...]{}
\setkeys{Gin}{width=\maxwidth,height=\maxheight,keepaspectratio}
\IfFileExists{parskip.sty}{%
\usepackage{parskip}
}{% else
\setlength{\parindent}{0pt}
\setlength{\parskip}{6pt plus 2pt minus 1pt}
}
\setlength{\emergencystretch}{3em}  % prevent overfull lines
\providecommand{\tightlist}{%
  \setlength{\itemsep}{0pt}\setlength{\parskip}{0pt}}
\setcounter{secnumdepth}{0}
% Redefines (sub)paragraphs to behave more like sections
\ifx\paragraph\undefined\else
\let\oldparagraph\paragraph
\renewcommand{\paragraph}[1]{\oldparagraph{#1}\mbox{}}
\fi
\ifx\subparagraph\undefined\else
\let\oldsubparagraph\subparagraph
\renewcommand{\subparagraph}[1]{\oldsubparagraph{#1}\mbox{}}
\fi

%%% Use protect on footnotes to avoid problems with footnotes in titles
\let\rmarkdownfootnote\footnote%
\def\footnote{\protect\rmarkdownfootnote}

%%% Change title format to be more compact
\usepackage{titling}

% Create subtitle command for use in maketitle
\newcommand{\subtitle}[1]{
  \posttitle{
    \begin{center}\large#1\end{center}
    }
}

\setlength{\droptitle}{-2em}

  \title{R for OLS}
    \pretitle{\vspace{\droptitle}\centering\huge}
  \posttitle{\par}
    \author{lin-tingyu}
    \preauthor{\centering\large\emph}
  \postauthor{\par}
      \predate{\centering\large\emph}
  \postdate{\par}
    \date{3/6/2019}


\begin{document}
\maketitle

{
\setcounter{tocdepth}{3}
\tableofcontents
}
\section{參考資料}

\begin{itemize}
\item
  \href{https://bookdown.org/PoMingChen/Dplyr_minicourse/}{dplyr
  minicourse, 陳柏銘}
\item
  R magrittr 套件:在 R 中使用管線(Pipe)處理資料流 - G. T. Wang.
  (2016). G. T. Wang. Retrieved 5 March 2019, from
  \url{https://blog.gtwang.org/r/r-pipes-magrittr-package/}
\end{itemize}

\section{setup}\label{setup}

\begin{Shaded}
\begin{Highlighting}[]
\KeywordTok{library}\NormalTok{(}\StringTok{"AER"}\NormalTok{)}
\KeywordTok{library}\NormalTok{(}\StringTok{"ggplot2"}\NormalTok{)}
\KeywordTok{library}\NormalTok{(}\StringTok{"dplyr"}\NormalTok{)}
\KeywordTok{library}\NormalTok{(}\StringTok{"knitr"}\NormalTok{)}
\end{Highlighting}
\end{Shaded}

\section{dataframe物件}\label{dataframe}

\begin{Shaded}
\begin{Highlighting}[]
\KeywordTok{data}\NormalTok{(}\StringTok{"Journals"}\NormalTok{)}
\end{Highlighting}
\end{Shaded}

\begin{quote}
Journal這個dataframe的結構(structure)是什麼?有幾個變數?每個變數物件的類別(class)又是什麼?
\end{quote}

\begin{quote}
找出Journal資料的詳細說明。
\end{quote}

\section{資料處理:產生新變數 dplyr::mutate}\label{-dplyrmutate}

\begin{Shaded}
\begin{Highlighting}[]
\CommentTok{#老師的寫法}
\NormalTok{ Journals }\OperatorTok\StringTok{ }\KeywordTok{mutate}\NormalTok{(}\DataTypeTok{citeprice=}\NormalTok{price}\OperatorTok{/}\NormalTok{citations) ->journals}
\KeywordTok{summary}\NormalTok{(journals)}
\CommentTok{# pip %>% 可以將第一位提到前面(在此將Journals往前提)}
\CommentTok{# mutate 可以將部分資料提出}
\end{Highlighting}
\end{Shaded}

\begin{Shaded}
\begin{Highlighting}[]
\CommentTok{# if follow usage}
\KeywordTok{mutate}\NormalTok{(Journals,}\DataTypeTok{citeprice=}\NormalTok{price}\OperatorTok{/}\NormalTok{citations)->journals}
\end{Highlighting}
\end{Shaded}

\section{因果問句}

\begin{quote}
期刊的價格(citeprice,平均文獻引用價格)如何影響其圖書館訂閱量(subs)?
\end{quote}

\begin{Shaded}
\begin{Highlighting}[]
\CommentTok{# 判斷變數是否為數值類別}
\NormalTok{is_numeric<-}\ControlFlowTok{function}\NormalTok{(x) }\KeywordTok{all}\NormalTok{(}\KeywordTok{is.numeric}\NormalTok{(x))}
\CommentTok{# 計算數數與citeprice的相關係數}
\NormalTok{cor_citeprice<-}\ControlFlowTok{function}\NormalTok{(x) }\KeywordTok{cor}\NormalTok{(x,journals}\OperatorTok{$}\NormalTok{citeprice)}

\NormalTok{journals }\OperatorTok\StringTok{  }
\StringTok{  }\KeywordTok{select_if}\NormalTok{(is_numeric) }\OperatorTok
\StringTok{  }\KeywordTok{summarise_all}\NormalTok{(cor_citeprice) }\OperatorTok
\StringTok{  }\KeywordTok{kable}\NormalTok{()}
\end{Highlighting}
\end{Shaded}

\begin{longtable}[]{@{}rrrrrrr@{}}
\toprule
price & pages & charpp & citations & foundingyear & subs &
citeprice\tabularnewline
\midrule
\endhead
0.1375851 & -0.268033 & 0.0003749 & -0.3020881 & 0.3102098 & -0.4195314
& 1\tabularnewline
\bottomrule
\end{longtable}

\begin{Shaded}
\begin{Highlighting}[]
\NormalTok{journals }\OperatorTok\StringTok{ }
\StringTok{  }\KeywordTok{lm}\NormalTok{(}\KeywordTok{log}\NormalTok{(subs)}\OperatorTok{~}\KeywordTok{log}\NormalTok{(citeprice),}\DataTypeTok{data=}\NormalTok{.) ->}\StringTok{ }\NormalTok{model1}

\NormalTok{journals }\OperatorTok
\StringTok{  }\KeywordTok{lm}\NormalTok{(}\KeywordTok{log}\NormalTok{(subs)}\OperatorTok{~}\KeywordTok{log}\NormalTok{(citeprice)}\OperatorTok{+}\NormalTok{foundingyear,}\DataTypeTok{data=}\NormalTok{.) ->}\StringTok{ }\NormalTok{model2}
\end{Highlighting}
\end{Shaded}

\begin{quote}
為什麼取log後,兩者的相關度變高?它表示兩個變數變得更不獨立嗎?
\end{quote}

\section{效應評估}

\begin{quote}
單純比較不同「期刊價格」(citeprice)的期刊所獨得的圖書館「訂閱數」(subs)變化並無法反應真正的「期刊價格」效應,原因是「立足點」並不與「期刊價格」獨立。
\end{quote}

\begin{quote}
這裡「立足點」指得是什麼?
\end{quote}

\section{進階關連分析}

數值變數v.s.數值變數

\begin{Shaded}
\begin{Highlighting}[]
\KeywordTok{library}\NormalTok{(broom)}
\KeywordTok{tidy}\NormalTok{(model1)}
\end{Highlighting}
\end{Shaded}

\begin{quote}
期刊越重要,其引用次數越高,因此高引用次數的期刊,你認為它在「低價格下的訂閱數」(立足點)會比較高還是低?
\end{quote}

\begin{quote}
承上題,單純比較「期刊引用單價」高低間的「訂閱數量」差別,所估算出來的價格效果以絕對值來看會高估、還是低估?為什麼?
\end{quote}

\section{複迴歸模型}

\subsection{模型比較}

\begin{Shaded}
\begin{Highlighting}[]
\NormalTok{journals }\OperatorTok\StringTok{ }
\StringTok{  }\KeywordTok{lm}\NormalTok{(}\KeywordTok{log}\NormalTok{(subs)}\OperatorTok{~}\KeywordTok{log}\NormalTok{(citeprice),}\DataTypeTok{data=}\NormalTok{.) ->}\StringTok{ }\NormalTok{model_}\DecValTok{1}
\NormalTok{journals }\OperatorTok
\StringTok{  }\KeywordTok{lm}\NormalTok{(}\KeywordTok{log}\NormalTok{(subs)}\OperatorTok{~}\KeywordTok{log}\NormalTok{(citeprice)}\OperatorTok{+}\NormalTok{foundingyear,}\DataTypeTok{data=}\NormalTok{.) ->}\StringTok{ }\NormalTok{model_}\DecValTok{2}

\KeywordTok{library}\NormalTok{(sandwich)}
\KeywordTok{library}\NormalTok{(lmtest)}
\KeywordTok{library}\NormalTok{(stargazer)}

\CommentTok{#使用vcovHC函數來計算HC1型的異質變異(即橫斷面資料下的線性迴歸模型)}
\KeywordTok{coeftest}\NormalTok{(model_}\DecValTok{1}\NormalTok{, }\DataTypeTok{vcov. =}\NormalTok{ vcovHC, }\DataTypeTok{type=}\StringTok{"HC1"}\NormalTok{) ->}\StringTok{ }\NormalTok{model_1_coeftest}
\KeywordTok{coeftest}\NormalTok{(model_}\DecValTok{2}\NormalTok{, }\DataTypeTok{vcov. =}\NormalTok{ vcovHC, }\DataTypeTok{type=}\StringTok{"HC1"}\NormalTok{) ->}\StringTok{ }\NormalTok{model_2_coeftest}

\KeywordTok{stargazer}\NormalTok{(model_}\DecValTok{1}\NormalTok{, model_}\DecValTok{2}\NormalTok{, }
          \DataTypeTok{se=}\KeywordTok{list}\NormalTok{(model_1_coeftest[,}\StringTok{"Std. Error"}\NormalTok{], model_2_coeftest[,}\DecValTok{2}\NormalTok{]),}
          \DataTypeTok{type=}\StringTok{"html"}\NormalTok{,}
          \DataTypeTok{align=}\OtherTok{TRUE}\NormalTok{)}
\end{Highlighting}
\end{Shaded}

\begin{verbatim}
## 
## <table style="text-align:center"><tr><td colspan="3" style="border-bottom: 1px solid black"></td></tr><tr><td style="text-align:left"></td><td colspan="2"><em>Dependent variable:</em></td></tr>
## <tr><td></td><td colspan="2" style="border-bottom: 1px solid black"></td></tr>
## <tr><td style="text-align:left"></td><td colspan="2">log(subs)</td></tr>
## <tr><td style="text-align:left"></td><td>(1)</td><td>(2)</td></tr>
## <tr><td colspan="3" style="border-bottom: 1px solid black"></td></tr><tr><td style="text-align:left">log(citeprice)</td><td>-0.533<sup>***</sup></td><td>-0.513<sup>***</sup></td></tr>
## <tr><td style="text-align:left"></td><td>(0.034)</td><td>(0.042)</td></tr>
## <tr><td style="text-align:left"></td><td></td><td></td></tr>
## <tr><td style="text-align:left">foundingyear</td><td></td><td>-0.003</td></tr>
## <tr><td style="text-align:left"></td><td></td><td>(0.003)</td></tr>
## <tr><td style="text-align:left"></td><td></td><td></td></tr>
## <tr><td style="text-align:left">Constant</td><td>4.766<sup>***</sup></td><td>10.016</td></tr>
## <tr><td style="text-align:left"></td><td>(0.055)</td><td>(6.629)</td></tr>
## <tr><td style="text-align:left"></td><td></td><td></td></tr>
## <tr><td colspan="3" style="border-bottom: 1px solid black"></td></tr><tr><td style="text-align:left">Observations</td><td>180</td><td>180</td></tr>
## <tr><td style="text-align:left">R<sup>2</sup></td><td>0.557</td><td>0.560</td></tr>
## <tr><td style="text-align:left">Adjusted R<sup>2</sup></td><td>0.555</td><td>0.555</td></tr>
## <tr><td style="text-align:left">Residual Std. Error</td><td>0.750 (df = 178)</td><td>0.749 (df = 177)</td></tr>
## <tr><td style="text-align:left">F Statistic</td><td>224.037<sup>***</sup> (df = 1; 178)</td><td>112.728<sup>***</sup> (df = 2; 177)</td></tr>
## <tr><td colspan="3" style="border-bottom: 1px solid black"></td></tr><tr><td style="text-align:left"><em>Note:</em></td><td colspan="2" style="text-align:right"><sup>*</sup>p<0.1; <sup>**</sup>p<0.05; <sup>***</sup>p<0.01</td></tr>
## </table>
\end{verbatim}

\begin{Shaded}
\begin{Highlighting}[]
\CommentTok{#將 模型1,2}
\CommentTok{#type="html" 用網頁表示}
\end{Highlighting}
\end{Shaded}

??stargazer


\end{document}
